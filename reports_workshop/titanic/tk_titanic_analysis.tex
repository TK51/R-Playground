% Options for packages loaded elsewhere
\PassOptionsToPackage{unicode}{hyperref}
\PassOptionsToPackage{hyphens}{url}
%
\documentclass[
]{article}
\usepackage{amsmath,amssymb}
\usepackage{lmodern}
\usepackage{iftex}
\ifPDFTeX
  \usepackage[T1]{fontenc}
  \usepackage[utf8]{inputenc}
  \usepackage{textcomp} % provide euro and other symbols
\else % if luatex or xetex
  \usepackage{unicode-math}
  \defaultfontfeatures{Scale=MatchLowercase}
  \defaultfontfeatures[\rmfamily]{Ligatures=TeX,Scale=1}
\fi
% Use upquote if available, for straight quotes in verbatim environments
\IfFileExists{upquote.sty}{\usepackage{upquote}}{}
\IfFileExists{microtype.sty}{% use microtype if available
  \usepackage[]{microtype}
  \UseMicrotypeSet[protrusion]{basicmath} % disable protrusion for tt fonts
}{}
\makeatletter
\@ifundefined{KOMAClassName}{% if non-KOMA class
  \IfFileExists{parskip.sty}{%
    \usepackage{parskip}
  }{% else
    \setlength{\parindent}{0pt}
    \setlength{\parskip}{6pt plus 2pt minus 1pt}}
}{% if KOMA class
  \KOMAoptions{parskip=half}}
\makeatother
\usepackage{xcolor}
\usepackage[left = 1.5cm, right = 1cm, top = 1cm, bottom = 1cm, margin =
1.5cm]{geometry}
\usepackage{color}
\usepackage{fancyvrb}
\newcommand{\VerbBar}{|}
\newcommand{\VERB}{\Verb[commandchars=\\\{\}]}
\DefineVerbatimEnvironment{Highlighting}{Verbatim}{commandchars=\\\{\}}
% Add ',fontsize=\small' for more characters per line
\usepackage{framed}
\definecolor{shadecolor}{RGB}{248,248,248}
\newenvironment{Shaded}{\begin{snugshade}}{\end{snugshade}}
\newcommand{\AlertTok}[1]{\textcolor[rgb]{0.94,0.16,0.16}{#1}}
\newcommand{\AnnotationTok}[1]{\textcolor[rgb]{0.56,0.35,0.01}{\textbf{\textit{#1}}}}
\newcommand{\AttributeTok}[1]{\textcolor[rgb]{0.77,0.63,0.00}{#1}}
\newcommand{\BaseNTok}[1]{\textcolor[rgb]{0.00,0.00,0.81}{#1}}
\newcommand{\BuiltInTok}[1]{#1}
\newcommand{\CharTok}[1]{\textcolor[rgb]{0.31,0.60,0.02}{#1}}
\newcommand{\CommentTok}[1]{\textcolor[rgb]{0.56,0.35,0.01}{\textit{#1}}}
\newcommand{\CommentVarTok}[1]{\textcolor[rgb]{0.56,0.35,0.01}{\textbf{\textit{#1}}}}
\newcommand{\ConstantTok}[1]{\textcolor[rgb]{0.00,0.00,0.00}{#1}}
\newcommand{\ControlFlowTok}[1]{\textcolor[rgb]{0.13,0.29,0.53}{\textbf{#1}}}
\newcommand{\DataTypeTok}[1]{\textcolor[rgb]{0.13,0.29,0.53}{#1}}
\newcommand{\DecValTok}[1]{\textcolor[rgb]{0.00,0.00,0.81}{#1}}
\newcommand{\DocumentationTok}[1]{\textcolor[rgb]{0.56,0.35,0.01}{\textbf{\textit{#1}}}}
\newcommand{\ErrorTok}[1]{\textcolor[rgb]{0.64,0.00,0.00}{\textbf{#1}}}
\newcommand{\ExtensionTok}[1]{#1}
\newcommand{\FloatTok}[1]{\textcolor[rgb]{0.00,0.00,0.81}{#1}}
\newcommand{\FunctionTok}[1]{\textcolor[rgb]{0.00,0.00,0.00}{#1}}
\newcommand{\ImportTok}[1]{#1}
\newcommand{\InformationTok}[1]{\textcolor[rgb]{0.56,0.35,0.01}{\textbf{\textit{#1}}}}
\newcommand{\KeywordTok}[1]{\textcolor[rgb]{0.13,0.29,0.53}{\textbf{#1}}}
\newcommand{\NormalTok}[1]{#1}
\newcommand{\OperatorTok}[1]{\textcolor[rgb]{0.81,0.36,0.00}{\textbf{#1}}}
\newcommand{\OtherTok}[1]{\textcolor[rgb]{0.56,0.35,0.01}{#1}}
\newcommand{\PreprocessorTok}[1]{\textcolor[rgb]{0.56,0.35,0.01}{\textit{#1}}}
\newcommand{\RegionMarkerTok}[1]{#1}
\newcommand{\SpecialCharTok}[1]{\textcolor[rgb]{0.00,0.00,0.00}{#1}}
\newcommand{\SpecialStringTok}[1]{\textcolor[rgb]{0.31,0.60,0.02}{#1}}
\newcommand{\StringTok}[1]{\textcolor[rgb]{0.31,0.60,0.02}{#1}}
\newcommand{\VariableTok}[1]{\textcolor[rgb]{0.00,0.00,0.00}{#1}}
\newcommand{\VerbatimStringTok}[1]{\textcolor[rgb]{0.31,0.60,0.02}{#1}}
\newcommand{\WarningTok}[1]{\textcolor[rgb]{0.56,0.35,0.01}{\textbf{\textit{#1}}}}
\usepackage{longtable,booktabs,array}
\usepackage{calc} % for calculating minipage widths
% Correct order of tables after \paragraph or \subparagraph
\usepackage{etoolbox}
\makeatletter
\patchcmd\longtable{\par}{\if@noskipsec\mbox{}\fi\par}{}{}
\makeatother
% Allow footnotes in longtable head/foot
\IfFileExists{footnotehyper.sty}{\usepackage{footnotehyper}}{\usepackage{footnote}}
\makesavenoteenv{longtable}
\usepackage{graphicx}
\makeatletter
\def\maxwidth{\ifdim\Gin@nat@width>\linewidth\linewidth\else\Gin@nat@width\fi}
\def\maxheight{\ifdim\Gin@nat@height>\textheight\textheight\else\Gin@nat@height\fi}
\makeatother
% Scale images if necessary, so that they will not overflow the page
% margins by default, and it is still possible to overwrite the defaults
% using explicit options in \includegraphics[width, height, ...]{}
\setkeys{Gin}{width=\maxwidth,height=\maxheight,keepaspectratio}
% Set default figure placement to htbp
\makeatletter
\def\fps@figure{htbp}
\makeatother
\setlength{\emergencystretch}{3em} % prevent overfull lines
\providecommand{\tightlist}{%
  \setlength{\itemsep}{0pt}\setlength{\parskip}{0pt}}
\setcounter{secnumdepth}{-\maxdimen} % remove section numbering
\ifLuaTeX
  \usepackage{selnolig}  % disable illegal ligatures
\fi
\IfFileExists{bookmark.sty}{\usepackage{bookmark}}{\usepackage{hyperref}}
\IfFileExists{xurl.sty}{\usepackage{xurl}}{} % add URL line breaks if available
\urlstyle{same} % disable monospaced font for URLs
\hypersetup{
  pdftitle={Titanic Analysis},
  pdfauthor={Taras the Analyst},
  hidelinks,
  pdfcreator={LaTeX via pandoc}}

\title{Titanic Analysis}
\author{Taras the Analyst}
\date{2023-04-09}

\begin{document}
\maketitle

\hypertarget{introduction}{%
\subsection{Introduction}\label{introduction}}

This is the report produced from the
\href{https://www.kaggle.com/code/taraskhamardiuk/getting-the-titanic-started-with-r-3rd-playbook}{Kaggle
notebook} `Titanic Analysis' by Taras K. from 03/18/2023.

The original inspirational
\href{https://www.kaggle.com/code/hillabehar/titanic-analysis-with-r/report}{source}
is by Hilla Behar

In this analysis the following questions were asked:

\begin{enumerate}
\def\labelenumi{\arabic{enumi}.}
\tightlist
\item
  What is the relationship the features and a passenger's chance of
  survival.
\item
  Prediction of survival for the entire ship.
\end{enumerate}

Last update: 09/04/2023 (see the list of updates at the end of this
work)

\hypertarget{setting-the-environment}{%
\subsection{Setting the environment}\label{setting-the-environment}}

\hypertarget{packages}{%
\subsubsection{Packages}\label{packages}}

\begin{Shaded}
\begin{Highlighting}[]
\CommentTok{\# The following packages are to be used for the current analysis}
\FunctionTok{library}\NormalTok{(dplyr)         }\CommentTok{\# for data manipulation}
\FunctionTok{library}\NormalTok{(tidyverse)     }\CommentTok{\# for working operations}
\FunctionTok{library}\NormalTok{(ggplot2)       }\CommentTok{\# for data visualization}
\FunctionTok{library}\NormalTok{(GGally)        }\CommentTok{\# Extension to \textquotesingle{}ggplot2\textquotesingle{}}
\FunctionTok{library}\NormalTok{(rpart)         }\CommentTok{\# decision tree model package}
\FunctionTok{library}\NormalTok{(rpart.plot)    }\CommentTok{\# decision tree visualization package}
\FunctionTok{library}\NormalTok{(ggcorrplot)    }\CommentTok{\# to understand the correlation matrix}
\FunctionTok{library}\NormalTok{(randomForest)  }\CommentTok{\# planting the trees needs some methodology...:)}
\FunctionTok{library}\NormalTok{(pander)        }\CommentTok{\# to create pretty tables}
\FunctionTok{library}\NormalTok{(knitr)         }\CommentTok{\# to create pretty tables}
\FunctionTok{library}\NormalTok{(tinytex)       }\CommentTok{\# to use the features for file rendering to .pdf}
\end{Highlighting}
\end{Shaded}

\hypertarget{loading-the-data-sources}{%
\subsubsection{Loading the data
sources}\label{loading-the-data-sources}}

\begin{Shaded}
\begin{Highlighting}[]
\NormalTok{test }\OtherTok{\textless{}{-}} \FunctionTok{read.csv}\NormalTok{(}\StringTok{\textquotesingle{}./test.csv\textquotesingle{}}\NormalTok{,}\AttributeTok{stringsAsFactors =} \ConstantTok{FALSE}\NormalTok{)}
\NormalTok{train }\OtherTok{\textless{}{-}} \FunctionTok{read.csv}\NormalTok{(}\StringTok{\textquotesingle{}./train.csv\textquotesingle{}}\NormalTok{, }\AttributeTok{stringsAsFactors =} \ConstantTok{FALSE}\NormalTok{)}
\FunctionTok{dim}\NormalTok{(test)   }\CommentTok{\# check the test data frame dimensions}
\end{Highlighting}
\end{Shaded}

\begin{verbatim}
## [1] 418  11
\end{verbatim}

\begin{Shaded}
\begin{Highlighting}[]
\FunctionTok{dim}\NormalTok{(train)  }\CommentTok{\# check the train data frame dimensions}
\end{Highlighting}
\end{Shaded}

\begin{verbatim}
## [1] 891  12
\end{verbatim}

\hypertarget{data-elaboration}{%
\subsection{Data elaboration}\label{data-elaboration}}

\hypertarget{merging-both-datasets-into-a-consolidated-one}{%
\subsubsection{Merging both datasets into a consolidated
one*}\label{merging-both-datasets-into-a-consolidated-one}}

\textbf{bind\_rows()} is to be used, as \textbf{rbind()} doesn't work
here due to different number of columns in \emph{train} and \emph{test}

\begin{Shaded}
\begin{Highlighting}[]
\NormalTok{full }\OtherTok{\textless{}{-}} \FunctionTok{bind\_rows}\NormalTok{(train,test) }
\FunctionTok{dim}\NormalTok{(full)  }\CommentTok{\# check the resulted data frame dimensions}
\end{Highlighting}
\end{Shaded}

\begin{verbatim}
## [1] 1309   12
\end{verbatim}

\begin{Shaded}
\begin{Highlighting}[]
\FunctionTok{str}\NormalTok{(full)  }\CommentTok{\# check the resulted data frame structure}
\end{Highlighting}
\end{Shaded}

\begin{verbatim}
## 'data.frame':    1309 obs. of  12 variables:
##  $ PassengerId: int  1 2 3 4 5 6 7 8 9 10 ...
##  $ Survived   : int  0 1 1 1 0 0 0 0 1 1 ...
##  $ Pclass     : int  3 1 3 1 3 3 1 3 3 2 ...
##  $ Name       : chr  "Braund, Mr. Owen Harris" "Cumings, Mrs. John Bradley (Florence Briggs Thayer)" "Heikkinen, Miss. Laina" "Futrelle, Mrs. Jacques Heath (Lily May Peel)" ...
##  $ Sex        : chr  "male" "female" "female" "female" ...
##  $ Age        : num  22 38 26 35 35 NA 54 2 27 14 ...
##  $ SibSp      : int  1 1 0 1 0 0 0 3 0 1 ...
##  $ Parch      : int  0 0 0 0 0 0 0 1 2 0 ...
##  $ Ticket     : chr  "A/5 21171" "PC 17599" "STON/O2. 3101282" "113803" ...
##  $ Fare       : num  7.25 71.28 7.92 53.1 8.05 ...
##  $ Cabin      : chr  "" "C85" "" "C123" ...
##  $ Embarked   : chr  "S" "C" "S" "S" ...
\end{verbatim}

The data is to be checked for missing values

\begin{verbatim}
## [1] "Here is missing value check:"
\end{verbatim}

\begin{verbatim}
## PassengerId    Survived      Pclass        Name         Sex         Age 
##           0         418           0           0           0         263 
##       SibSp       Parch      Ticket        Fare       Cabin    Embarked 
##           0           0           0           1           0           0
\end{verbatim}

\begin{verbatim}
## PassengerId    Survived      Pclass        Name         Sex         Age 
##           0          NA           0           0           0          NA 
##       SibSp       Parch      Ticket        Fare       Cabin    Embarked 
##           0           0           0          NA        1014           2
\end{verbatim}

So, the ouput is: N/As - left table, NULLs - right table

\begin{Shaded}
\begin{Highlighting}[]
\NormalTok{knitr}\SpecialCharTok{::}\FunctionTok{kable}\NormalTok{(}\FunctionTok{list}\NormalTok{(k1, k2))}
\end{Highlighting}
\end{Shaded}

\begin{table}

\centering
\begin{tabular}[t]{l|r}
\hline
  & x\\
\hline
PassengerId & 0\\
\hline
Survived & 418\\
\hline
Pclass & 0\\
\hline
Name & 0\\
\hline
Sex & 0\\
\hline
Age & 263\\
\hline
SibSp & 0\\
\hline
Parch & 0\\
\hline
Ticket & 0\\
\hline
Fare & 1\\
\hline
Cabin & 0\\
\hline
Embarked & 0\\
\hline
\end{tabular}
\centering
\begin{tabular}[t]{l|r}
\hline
  & x\\
\hline
PassengerId & 0\\
\hline
Survived & NA\\
\hline
Pclass & 0\\
\hline
Name & 0\\
\hline
Sex & 0\\
\hline
Age & NA\\
\hline
SibSp & 0\\
\hline
Parch & 0\\
\hline
Ticket & 0\\
\hline
Fare & NA\\
\hline
Cabin & 1014\\
\hline
Embarked & 2\\
\hline
\end{tabular}
\end{table}

\begin{Shaded}
\begin{Highlighting}[]
\CommentTok{\# cross{-}checking the empty records for Embarked}
\FunctionTok{filter}\NormalTok{(full, full}\SpecialCharTok{$}\NormalTok{Embarked }\SpecialCharTok{==} \StringTok{""}\NormalTok{)}
\end{Highlighting}
\end{Shaded}

\begin{verbatim}
##   PassengerId Survived Pclass                                      Name    Sex
## 1          62        1      1                       Icard, Miss. Amelie female
## 2         830        1      1 Stone, Mrs. George Nelson (Martha Evelyn) female
##   Age SibSp Parch Ticket Fare Cabin Embarked
## 1  38     0     0 113572   80   B28         
## 2  62     0     0 113572   80   B28
\end{verbatim}

\begin{Shaded}
\begin{Highlighting}[]
\CommentTok{\# getting it into a bit more visually attractive way}
\FunctionTok{kable}\NormalTok{(}\FunctionTok{filter}\NormalTok{(full, full}\SpecialCharTok{$}\NormalTok{Embarked }\SpecialCharTok{==} \StringTok{""}\NormalTok{))}
\end{Highlighting}
\end{Shaded}

\begin{longtable}[]{@{}
  >{\raggedleft\arraybackslash}p{(\columnwidth - 22\tabcolsep) * \real{0.1000}}
  >{\raggedleft\arraybackslash}p{(\columnwidth - 22\tabcolsep) * \real{0.0750}}
  >{\raggedleft\arraybackslash}p{(\columnwidth - 22\tabcolsep) * \real{0.0583}}
  >{\raggedright\arraybackslash}p{(\columnwidth - 22\tabcolsep) * \real{0.3500}}
  >{\raggedright\arraybackslash}p{(\columnwidth - 22\tabcolsep) * \real{0.0583}}
  >{\raggedleft\arraybackslash}p{(\columnwidth - 22\tabcolsep) * \real{0.0333}}
  >{\raggedleft\arraybackslash}p{(\columnwidth - 22\tabcolsep) * \real{0.0500}}
  >{\raggedleft\arraybackslash}p{(\columnwidth - 22\tabcolsep) * \real{0.0500}}
  >{\raggedright\arraybackslash}p{(\columnwidth - 22\tabcolsep) * \real{0.0583}}
  >{\raggedleft\arraybackslash}p{(\columnwidth - 22\tabcolsep) * \real{0.0417}}
  >{\raggedright\arraybackslash}p{(\columnwidth - 22\tabcolsep) * \real{0.0500}}
  >{\raggedright\arraybackslash}p{(\columnwidth - 22\tabcolsep) * \real{0.0750}}@{}}
\toprule()
\begin{minipage}[b]{\linewidth}\raggedleft
PassengerId
\end{minipage} & \begin{minipage}[b]{\linewidth}\raggedleft
Survived
\end{minipage} & \begin{minipage}[b]{\linewidth}\raggedleft
Pclass
\end{minipage} & \begin{minipage}[b]{\linewidth}\raggedright
Name
\end{minipage} & \begin{minipage}[b]{\linewidth}\raggedright
Sex
\end{minipage} & \begin{minipage}[b]{\linewidth}\raggedleft
Age
\end{minipage} & \begin{minipage}[b]{\linewidth}\raggedleft
SibSp
\end{minipage} & \begin{minipage}[b]{\linewidth}\raggedleft
Parch
\end{minipage} & \begin{minipage}[b]{\linewidth}\raggedright
Ticket
\end{minipage} & \begin{minipage}[b]{\linewidth}\raggedleft
Fare
\end{minipage} & \begin{minipage}[b]{\linewidth}\raggedright
Cabin
\end{minipage} & \begin{minipage}[b]{\linewidth}\raggedright
Embarked
\end{minipage} \\
\midrule()
\endhead
62 & 1 & 1 & Icard, Miss. Amelie & female & 38 & 0 & 0 & 113572 & 80 &
B28 & \\
830 & 1 & 1 & Stone, Mrs.~George Nelson (Martha Evelyn) & female & 62 &
0 & 0 & 113572 & 80 & B28 & \\
\bottomrule()
\end{longtable}

\begin{Shaded}
\begin{Highlighting}[]
\CommentTok{\# getting the digits of missing values}
\FunctionTok{paste}\NormalTok{(}\StringTok{"= N/A in full dataset:"}\NormalTok{)  }\CommentTok{\# that\textquotesingle{}s added for some internal explanations}
\end{Highlighting}
\end{Shaded}

\begin{verbatim}
## [1] "= N/A in full dataset:"
\end{verbatim}

\begin{Shaded}
\begin{Highlighting}[]
\FunctionTok{pander}\NormalTok{(}\FunctionTok{table}\NormalTok{(}\FunctionTok{is.na}\NormalTok{(full)))  }\CommentTok{\# showing aggregated "n/a" values within each column}
\end{Highlighting}
\end{Shaded}

\begin{longtable}[]{@{}
  >{\centering\arraybackslash}p{(\columnwidth - 2\tabcolsep) * \real{0.1111}}
  >{\centering\arraybackslash}p{(\columnwidth - 2\tabcolsep) * \real{0.1111}}@{}}
\toprule()
\begin{minipage}[b]{\linewidth}\centering
FALSE
\end{minipage} & \begin{minipage}[b]{\linewidth}\centering
TRUE
\end{minipage} \\
\midrule()
\endhead
15026 & 682 \\
\bottomrule()
\end{longtable}

\hypertarget{cleaning-transforming-the-data}{%
\subsection{Cleaning \& transforming the
data}\label{cleaning-transforming-the-data}}

\hypertarget{change-the-empty-strings-in-embarked-to-the-first-choice-c}{%
\paragraph{Change the empty strings in Embarked to the first choice
``C''}\label{change-the-empty-strings-in-embarked-to-the-first-choice-c}}

\begin{Shaded}
\begin{Highlighting}[]
\NormalTok{full}\SpecialCharTok{$}\NormalTok{Embarked[full}\SpecialCharTok{$}\NormalTok{Embarked }\SpecialCharTok{==} \StringTok{""}\NormalTok{] }\OtherTok{=} \StringTok{"C"}
\end{Highlighting}
\end{Shaded}

\hypertarget{see-how-many-features-can-be-transformed-to-factors}{%
\paragraph{See how many features can be transformed to
factors}\label{see-how-many-features-can-be-transformed-to-factors}}

\begin{Shaded}
\begin{Highlighting}[]
\FunctionTok{apply}\NormalTok{(full, }\DecValTok{2}\NormalTok{, }\ControlFlowTok{function}\NormalTok{(x) }\FunctionTok{length}\NormalTok{(}\FunctionTok{unique}\NormalTok{(x)))}
\end{Highlighting}
\end{Shaded}

\begin{verbatim}
## PassengerId    Survived      Pclass        Name         Sex         Age 
##        1309           3           3        1307           2          99 
##       SibSp       Parch      Ticket        Fare       Cabin    Embarked 
##           7           8         929         282         187           3
\end{verbatim}

Move the attributes Survived, Pclass, Sex, Embarked to be factors

\begin{Shaded}
\begin{Highlighting}[]
\NormalTok{cols }\OtherTok{\textless{}{-}} \FunctionTok{as.factor}\NormalTok{(}\FunctionTok{c}\NormalTok{(}\StringTok{"Survived"}\NormalTok{, }\StringTok{"Pclass"}\NormalTok{, }\StringTok{"Sex"}\NormalTok{, }\StringTok{"Embarked"}\NormalTok{))}
\ControlFlowTok{for}\NormalTok{ (i }\ControlFlowTok{in}\NormalTok{ cols)\{}
\NormalTok{  full[, i] }\OtherTok{\textless{}{-}} \FunctionTok{as.factor}\NormalTok{(full[, i])}
\NormalTok{\}}
\end{Highlighting}
\end{Shaded}

\hypertarget{now-lets-look-on-the-structure-of-the-full-data-set}{%
\paragraph{Now let's look on the structure of the full data
set}\label{now-lets-look-on-the-structure-of-the-full-data-set}}

\begin{Shaded}
\begin{Highlighting}[]
\FunctionTok{str}\NormalTok{(full)}
\end{Highlighting}
\end{Shaded}

\begin{verbatim}
## 'data.frame':    1309 obs. of  12 variables:
##  $ PassengerId: int  1 2 3 4 5 6 7 8 9 10 ...
##  $ Survived   : Factor w/ 2 levels "0","1": 1 2 2 2 1 1 1 1 2 2 ...
##  $ Pclass     : Factor w/ 3 levels "1","2","3": 3 1 3 1 3 3 1 3 3 2 ...
##  $ Name       : chr  "Braund, Mr. Owen Harris" "Cumings, Mrs. John Bradley (Florence Briggs Thayer)" "Heikkinen, Miss. Laina" "Futrelle, Mrs. Jacques Heath (Lily May Peel)" ...
##  $ Sex        : Factor w/ 2 levels "female","male": 2 1 1 1 2 2 2 2 1 1 ...
##  $ Age        : num  22 38 26 35 35 NA 54 2 27 14 ...
##  $ SibSp      : int  1 1 0 1 0 0 0 3 0 1 ...
##  $ Parch      : int  0 0 0 0 0 0 0 1 2 0 ...
##  $ Ticket     : chr  "A/5 21171" "PC 17599" "STON/O2. 3101282" "113803" ...
##  $ Fare       : num  7.25 71.28 7.92 53.1 8.05 ...
##  $ Cabin      : chr  "" "C85" "" "C123" ...
##  $ Embarked   : Factor w/ 3 levels "C","Q","S": 3 1 3 3 3 2 3 3 3 1 ...
\end{verbatim}

Move the attributes Survived, Pclass, Sex, Embarked to be factors within
train data set

\begin{Shaded}
\begin{Highlighting}[]
\NormalTok{cols }\OtherTok{\textless{}{-}} \FunctionTok{as.factor}\NormalTok{(}\FunctionTok{c}\NormalTok{(}\StringTok{"Survived"}\NormalTok{, }\StringTok{"Pclass"}\NormalTok{, }\StringTok{"Sex"}\NormalTok{, }\StringTok{"Embarked"}\NormalTok{))}
\ControlFlowTok{for}\NormalTok{ (i }\ControlFlowTok{in}\NormalTok{ cols)\{}
\NormalTok{  train[, i] }\OtherTok{\textless{}{-}} \FunctionTok{as.factor}\NormalTok{(train[, i])}
\NormalTok{\}}
\end{Highlighting}
\end{Shaded}

\hypertarget{now-lets-look-on-the-structure-of-the-train-data-set}{%
\paragraph{Now let's look on the structure of the train data
set}\label{now-lets-look-on-the-structure-of-the-train-data-set}}

str(train)

\hypertarget{analyse-the-cleaned-data}{%
\subsection{Analyse the cleaned data}\label{analyse-the-cleaned-data}}

The data has been loaded \& cleaned a little bit so far. Now, it's time
to look at the relationships between the different attributes within set
and to check the correlations within factored attributes, so to see if
there's something useful.

\begin{Shaded}
\begin{Highlighting}[]
\NormalTok{full\_fctrs }\OtherTok{\textless{}{-}}\NormalTok{ full[, }\FunctionTok{c}\NormalTok{(}\StringTok{"Survived"}\NormalTok{, }\StringTok{"Pclass"}\NormalTok{, }\StringTok{"Sex"}\NormalTok{, }\StringTok{"Embarked"}\NormalTok{)]}
\NormalTok{train\_fctrs }\OtherTok{\textless{}{-}}\NormalTok{ train[, }\FunctionTok{c}\NormalTok{(}\StringTok{"Survived"}\NormalTok{, }\StringTok{"Pclass"}\NormalTok{, }\StringTok{"Sex"}\NormalTok{, }\StringTok{"Embarked"}\NormalTok{)]}
\end{Highlighting}
\end{Shaded}

\begin{Shaded}
\begin{Highlighting}[]
\FunctionTok{dim}\NormalTok{(full\_fctrs)  }\CommentTok{\# check if the re{-}shaping went well resulted in 4 columns only}
\end{Highlighting}
\end{Shaded}

\begin{verbatim}
## [1] 1309    4
\end{verbatim}

\begin{Shaded}
\begin{Highlighting}[]
\FunctionTok{dim}\NormalTok{(train\_fctrs)}
\end{Highlighting}
\end{Shaded}

\begin{verbatim}
## [1] 891   4
\end{verbatim}

\begin{Shaded}
\begin{Highlighting}[]
\FunctionTok{str}\NormalTok{(train\_fctrs)  }\CommentTok{\# getting the structure overview of train factors}
\end{Highlighting}
\end{Shaded}

\begin{verbatim}
## 'data.frame':    891 obs. of  4 variables:
##  $ Survived: Factor w/ 2 levels "0","1": 1 2 2 2 1 1 1 1 2 2 ...
##  $ Pclass  : Factor w/ 3 levels "1","2","3": 3 1 3 1 3 3 1 3 3 2 ...
##  $ Sex     : Factor w/ 2 levels "female","male": 2 1 1 1 2 2 2 2 1 1 ...
##  $ Embarked: Factor w/ 4 levels "","C","Q","S": 4 2 4 4 4 3 4 4 4 2 ...
\end{verbatim}

\begin{Shaded}
\begin{Highlighting}[]
\FunctionTok{str}\NormalTok{(full\_fctrs)   }\CommentTok{\# getting the structure overview of test factors}
\end{Highlighting}
\end{Shaded}

\begin{verbatim}
## 'data.frame':    1309 obs. of  4 variables:
##  $ Survived: Factor w/ 2 levels "0","1": 1 2 2 2 1 1 1 1 2 2 ...
##  $ Pclass  : Factor w/ 3 levels "1","2","3": 3 1 3 1 3 3 1 3 3 2 ...
##  $ Sex     : Factor w/ 2 levels "female","male": 2 1 1 1 2 2 2 2 1 1 ...
##  $ Embarked: Factor w/ 3 levels "C","Q","S": 3 1 3 3 3 2 3 3 3 1 ...
\end{verbatim}

\hypertarget{adding-some-visuals-to-clarify-the-picture}{%
\paragraph{Adding some visuals to clarify the
picture}\label{adding-some-visuals-to-clarify-the-picture}}

\begin{Shaded}
\begin{Highlighting}[]
\CommentTok{\# visualizing the numbers available so to get the general picture of the case}
\FunctionTok{ggplot}\NormalTok{(}\AttributeTok{data =}\NormalTok{ train\_fctrs, }\FunctionTok{aes}\NormalTok{(}\AttributeTok{x =}\NormalTok{ Survived, }\AttributeTok{fill =}\NormalTok{ Sex)) }\SpecialCharTok{+} \FunctionTok{geom\_bar}\NormalTok{() }\SpecialCharTok{+}
  \FunctionTok{scale\_y\_continuous}\NormalTok{(}\AttributeTok{limits =} \FunctionTok{c}\NormalTok{(}\DecValTok{0}\NormalTok{, }\DecValTok{600}\NormalTok{)) }\SpecialCharTok{+}              \CommentTok{\# making visual limits}
  \FunctionTok{scale\_fill\_manual}\NormalTok{(}\AttributeTok{values =} \FunctionTok{c}\NormalTok{(}\StringTok{"\#c979a0"}\NormalTok{, }\StringTok{"\#79C9A2"}\NormalTok{)) }\SpecialCharTok{+} \CommentTok{\# color code for sex categories}
  \FunctionTok{labs}\NormalTok{(}\AttributeTok{title =} \StringTok{"Survived by Sex"}\NormalTok{,                       }\CommentTok{\# setting labels}
       \AttributeTok{subtitle =} \StringTok{"The Titanic Case"}\NormalTok{,}
       \AttributeTok{caption =} \StringTok{"Data from the Titanic dataset"}\NormalTok{, }
       \AttributeTok{x =} \StringTok{"}\SpecialCharTok{\textbackslash{}n}\StringTok{ Survived"}\NormalTok{, }\AttributeTok{y =} \StringTok{"Persons }\SpecialCharTok{\textbackslash{}n}\StringTok{"}\NormalTok{) }\SpecialCharTok{+}           \CommentTok{\# placing extra{-}line for better readability}
  \FunctionTok{theme}\NormalTok{(}\AttributeTok{axis.text =} \FunctionTok{element\_text}\NormalTok{(}\AttributeTok{size =} \DecValTok{12}\NormalTok{), }
        \AttributeTok{axis.title =} \FunctionTok{element\_text}\NormalTok{(}\AttributeTok{size =} \DecValTok{12}\NormalTok{, }\AttributeTok{face =} \StringTok{"bold"}\NormalTok{), }
        \AttributeTok{plot.title =} \FunctionTok{element\_text}\NormalTok{(}\AttributeTok{size =} \DecValTok{14}\NormalTok{, }\AttributeTok{hjust =} \FloatTok{0.5}\NormalTok{, }\AttributeTok{face =} \StringTok{"bold"}\NormalTok{))}
\end{Highlighting}
\end{Shaded}

\includegraphics{tk_titanic_analysis_files/figure-latex/unnamed-chunk-18-1.pdf}

\begin{Shaded}
\begin{Highlighting}[]
\CommentTok{\# side{-}by{-}side comparison to make things more understandable {-} Survived by Sex}
\FunctionTok{ggplot}\NormalTok{(}\AttributeTok{data =}\NormalTok{ train\_fctrs, }\FunctionTok{aes}\NormalTok{(}\AttributeTok{x =}\NormalTok{ Survived, }\AttributeTok{fill =}\NormalTok{ Sex)) }\SpecialCharTok{+} \FunctionTok{geom\_bar}\NormalTok{(}\AttributeTok{position =} \StringTok{"dodge"}\NormalTok{) }\SpecialCharTok{+} 
  \FunctionTok{scale\_y\_continuous}\NormalTok{(}\AttributeTok{limits =} \FunctionTok{c}\NormalTok{(}\DecValTok{0}\NormalTok{, }\DecValTok{600}\NormalTok{)) }\SpecialCharTok{+}                \CommentTok{\# making visual limits}
  \FunctionTok{scale\_fill\_manual}\NormalTok{(}\AttributeTok{values =} \FunctionTok{c}\NormalTok{(}\StringTok{"\#c979a0"}\NormalTok{, }\StringTok{"\#79C9A2"}\NormalTok{)) }\SpecialCharTok{+}   \CommentTok{\# color code for sex categories}
  \FunctionTok{labs}\NormalTok{(}\AttributeTok{title =} \StringTok{"Survived by Sex"}\NormalTok{,                         }\CommentTok{\# setting labels}
       \AttributeTok{subtitle =} \StringTok{"The Titanic Case"}\NormalTok{,}
       \AttributeTok{caption =} \StringTok{"Data from the Titanic dataset"}\NormalTok{, }
       \AttributeTok{x =} \StringTok{"}\SpecialCharTok{\textbackslash{}n}\StringTok{ Survived"}\NormalTok{, }\AttributeTok{y =} \StringTok{"Persons }\SpecialCharTok{\textbackslash{}n}\StringTok{"}\NormalTok{) }\SpecialCharTok{+} 
  \FunctionTok{theme}\NormalTok{(}\AttributeTok{axis.text =} \FunctionTok{element\_text}\NormalTok{(}\AttributeTok{size =} \DecValTok{12}\NormalTok{), }
        \AttributeTok{axis.title =} \FunctionTok{element\_text}\NormalTok{(}\AttributeTok{size =} \DecValTok{12}\NormalTok{, }\AttributeTok{face =} \StringTok{"bold"}\NormalTok{), }
        \AttributeTok{plot.title =} \FunctionTok{element\_text}\NormalTok{(}\AttributeTok{size =} \DecValTok{14}\NormalTok{, }\AttributeTok{hjust =} \FloatTok{0.5}\NormalTok{, }\AttributeTok{face =} \StringTok{"bold"}\NormalTok{))}
\end{Highlighting}
\end{Shaded}

\includegraphics{tk_titanic_analysis_files/figure-latex/unnamed-chunk-19-1.pdf}

\begin{Shaded}
\begin{Highlighting}[]
\CommentTok{\# side{-}by{-}side comparison to make things more understandable {-} Survived by Sex}
\FunctionTok{ggplot}\NormalTok{(}\AttributeTok{data =}\NormalTok{ train\_fctrs, }\FunctionTok{aes}\NormalTok{(}\AttributeTok{x =}\NormalTok{ Pclass, }\AttributeTok{fill =}\NormalTok{ Survived)) }\SpecialCharTok{+} \FunctionTok{geom\_bar}\NormalTok{() }\SpecialCharTok{+}
  \FunctionTok{scale\_y\_continuous}\NormalTok{(}\AttributeTok{limits =} \FunctionTok{c}\NormalTok{(}\DecValTok{0}\NormalTok{, }\DecValTok{600}\NormalTok{)) }\SpecialCharTok{+}                \CommentTok{\# making visual limits}
  \FunctionTok{scale\_fill\_manual}\NormalTok{(}\AttributeTok{values =} \FunctionTok{c}\NormalTok{(}\StringTok{"\#c979a0"}\NormalTok{, }\StringTok{"\#79C9A2"}\NormalTok{)) }\SpecialCharTok{+}   \CommentTok{\# color code for sex categories}
  \FunctionTok{labs}\NormalTok{(}\AttributeTok{title =} \StringTok{"Survived by Class"}\NormalTok{, }
       \AttributeTok{subtitle =} \StringTok{"The Titanic Case"}\NormalTok{,}
       \AttributeTok{caption =} \StringTok{"Data from the Titanic dataset"}\NormalTok{, }
       \AttributeTok{x =} \StringTok{"}\SpecialCharTok{\textbackslash{}n}\StringTok{ Class"}\NormalTok{, }\AttributeTok{y =} \StringTok{"Persons }\SpecialCharTok{\textbackslash{}n}\StringTok{"}\NormalTok{)}\SpecialCharTok{+} 
  \FunctionTok{theme}\NormalTok{(}\AttributeTok{axis.text =} \FunctionTok{element\_text}\NormalTok{(}\AttributeTok{size =} \DecValTok{12}\NormalTok{), }
        \AttributeTok{axis.title =} \FunctionTok{element\_text}\NormalTok{(}\AttributeTok{size =} \DecValTok{12}\NormalTok{, }\AttributeTok{face =} \StringTok{"bold"}\NormalTok{), }
        \AttributeTok{plot.title =} \FunctionTok{element\_text}\NormalTok{(}\AttributeTok{size =} \DecValTok{14}\NormalTok{, }\AttributeTok{hjust =} \FloatTok{0.5}\NormalTok{, }\AttributeTok{face =} \StringTok{"bold"}\NormalTok{))}
\end{Highlighting}
\end{Shaded}

\includegraphics{tk_titanic_analysis_files/figure-latex/unnamed-chunk-20-1.pdf}

\begin{Shaded}
\begin{Highlighting}[]
\CommentTok{\# side{-}by{-}side comparison to make things more understandable {-} Survived by Class}
\FunctionTok{ggplot}\NormalTok{(}\AttributeTok{data =}\NormalTok{ train\_fctrs, }\FunctionTok{aes}\NormalTok{(}\AttributeTok{x =}\NormalTok{ Pclass, }\AttributeTok{fill =}\NormalTok{ Survived)) }\SpecialCharTok{+} \FunctionTok{geom\_bar}\NormalTok{(}\AttributeTok{position =} \StringTok{"dodge"}\NormalTok{) }\SpecialCharTok{+}
  \FunctionTok{scale\_y\_continuous}\NormalTok{(}\AttributeTok{limits =} \FunctionTok{c}\NormalTok{(}\DecValTok{0}\NormalTok{, }\DecValTok{400}\NormalTok{)) }\SpecialCharTok{+}
  \FunctionTok{scale\_fill\_manual}\NormalTok{(}\AttributeTok{values =} \FunctionTok{c}\NormalTok{(}\StringTok{"\#a79da1"}\NormalTok{, }\StringTok{"\#79C9A2"}\NormalTok{)) }\SpecialCharTok{+}   \CommentTok{\# color code for sex categories}
  \FunctionTok{labs}\NormalTok{(}\AttributeTok{title =} \StringTok{"Survived by Class"}\NormalTok{, }
       \AttributeTok{subtitle =} \StringTok{"The Titanic Case"}\NormalTok{,}
       \AttributeTok{caption =} \StringTok{"Data from the Titanic dataset"}\NormalTok{, }
       \AttributeTok{x =} \StringTok{"}\SpecialCharTok{\textbackslash{}n}\StringTok{ Class"}\NormalTok{, }\AttributeTok{y =} \StringTok{"Persons }\SpecialCharTok{\textbackslash{}n}\StringTok{"}\NormalTok{)}\SpecialCharTok{+} 
  \FunctionTok{theme}\NormalTok{(}\AttributeTok{axis.text =} \FunctionTok{element\_text}\NormalTok{(}\AttributeTok{size =} \DecValTok{12}\NormalTok{), }
        \AttributeTok{axis.title =} \FunctionTok{element\_text}\NormalTok{(}\AttributeTok{size =} \DecValTok{12}\NormalTok{, }\AttributeTok{face =} \StringTok{"bold"}\NormalTok{), }
        \AttributeTok{plot.title =} \FunctionTok{element\_text}\NormalTok{(}\AttributeTok{size =} \DecValTok{14}\NormalTok{, }\AttributeTok{hjust =} \FloatTok{0.5}\NormalTok{, }\AttributeTok{face =} \StringTok{"bold"}\NormalTok{))}
\end{Highlighting}
\end{Shaded}

\includegraphics{tk_titanic_analysis_files/figure-latex/unnamed-chunk-21-1.pdf}

\hypertarget{create-a-train-limited-dataset-so-the-data-from-train-is-only-considered-in-calculations}{%
\section{create a train-limited dataset, so the data from train is only
considered in
calculations}\label{create-a-train-limited-dataset-so-the-data-from-train-is-only-considered-in-calculations}}

LT = dim(train){[}1{]} MT = dim(test){[}1{]} \# {[}1{]} stands for
number of rows, {[}2{]} stands for number of columns within the data
frame LT \# train dataset limited - gives number of rows to be used for
certain purposes, like limited calculations MT \# test dataset limited -
\ldots{} \#\# check the relationship between Sex and Survival:
ggplot(data = full{[}1:LT,{]},aes(x = Sex,fill = Survived)) +
geom\_bar() + \# {[}1:LT{]} = look into train data, without N/As in test
scale\_fill\_manual(values = c(``\#a79da1'', ``\#79C9A2'')) + \# color
code for sex categories labs(title = ``Survived by Sex'', subtitle =
``The Titanic Case'', caption = ``Data from the Titanic dataset'', x =
``\n Survived'', y = ``Persons \n'') \# Survival as a function of
Embarked: ggplot(data = full{[}1:LT,{]}, aes(x = Embarked, fill =
Survived)) + geom\_bar(position = ``fill'') + scale\_fill\_manual(values
= c(``\#a79da1'', ``\#79C9A2'')) + \# color code for sex categories
labs(title = ``Survived Proportions by Embarked'', subtitle = ``The
Titanic Case'', caption = ``Data from the Titanic dataset'', x =
``\n Embarked'', y = ``Frequency \n'') \# get the numbers of Survived
within Embarked classes, =1 - survived, =0 - didn't survive
t\textless-table(full{[}1:LT,{]}\(Embarked,full[1:LT,]\)Survived) t
addmargins(table(full{[}1:LT,{]}\(Embarked,full[1:LT,]\)Survived)) \#
get the percentage of Survived within Embarked classes, =1 - survived,
=0 - didn't survive
t\textless-table(full{[}1:LT,{]}\(Embarked,full[1:LT,]\)Survived) for (i
in 1:dim(t){[}1{]})\{ t{[}i,{]}\textless-t{[}i,{]}/sum(t{[}i,{]})*100 \}
t \# It looks like chances for survival were higher for those Embarked
in `C' (55\% compared to 33\% and 38\%, row-wise). \# But it is a bit
skewed, if to compare the number of victims and produce column-wise
ratio calculations \# Survival as a function of Pclass: ggplot(data =
full{[}1:LT,{]},aes(x=Pclass,fill=Survived)) + geom\_bar(position =
``fill'') + scale\_fill\_manual(values = c(``\#a79da1'', ``\#79C9A2''))
+ \# color code for sex categories labs(title = ``Survived Proportions
by Passenger Class'', subtitle = ``The Titanic Case'', caption = ``Data
from the Titanic dataset'', x = ``\n Pclass'', y = ``Frequency \n'') \#
It looks like you have a better chance to survive if you were in lower
ticket class. \# check the of Embarked versus Pclass: ggplot(data =
full{[}1:LT,{]},aes(x=Embarked,fill=Survived))+geom\_bar(position=``fill'')+facet\_wrap(\textasciitilde Pclass)
+ scale\_fill\_manual(values = c(``\#a79da1'', ``\#79C9A2'')) + \# color
code for sex categories labs(title = ``Survived Proportions by Passenger
Class vs Embarked Type'', subtitle = ``The Titanic Case'', caption =
``Data from the Titanic dataset'', x = ``\n Embarked'', y = ``Frequency
\n'') \# Now it's not so clear that there is a correlation between
Embarked and Survival.

\hypertarget{survivial-as-a-function-of-sibsp}{%
\section{Survivial as a function of
SibSp}\label{survivial-as-a-function-of-sibsp}}

ggplot(data = full{[}1:LT,{]},aes(x=SibSp,fill=Survived))+geom\_bar()+
scale\_fill\_manual(values = c(``\#a79da1'', ``\#79C9A2'')) + \# color
code for sex categories labs(title = ``Survived by SibSp'', subtitle =
``The Titanic Case'', caption = ``Data from the Titanic dataset'', x =
``\n SibSp'', y = ``Persons \n'') \# Survivial as a function of Parch
ggplot(data = full{[}1:LT,{]},aes(x=Parch,fill=Survived))+geom\_bar()+
scale\_fill\_manual(values = c(``\#a79da1'', ``\#79C9A2'')) + \# color
code for sex categories labs(title = ``Survived by Parch'', subtitle =
``The Titanic Case'', caption = ``Data from the Titanic dataset'', x =
``\n Parch'', y = ``Persons \n'') \# the dynamics of both attributes -
SibSp and Parch - seem to be quite similar \# check the attribute of
family size.

full\(FamilySize <- full\)SibSp +
full\(Parch +1; full1<-full[1:LT,] ggplot(data = full1[!is.na(full[1:LT,]\)FamilySize),{]},aes(x=FamilySize,fill=Survived))+geom\_histogram(binwidth
=1,position=``fill'')+ scale\_fill\_manual(values = c(``\#a79da1'',
``\#79C9A2'')) + \# color code for sex categories labs(title = ``Family
Size Survival Specifics'', subtitle = ``The Titanic Case'', caption =
``Data from the Titanic dataset'', x = ``\n Family Members'', y =
``Frequency \n'')+ theme(axis.text = element\_text(size = 12),
axis.title = element\_text(size = 12, face = ``bold''), plot.title =
element\_text(size = 14, hjust = 0.5, face = ``bold'')) \# That shows
that families with a family size bigger or equal to 2 but less than 6
have a more than 50\% to survive, \# in contrast to families with 1
member or more than 5 members. \# Survival as a function of age:

ggplot(data =
full1{[}!(is.na(full{[}1:LT,{]}\(Age)),],aes(x=Age,fill=Survived))+geom_histogram(binwidth =3)+  scale_fill_manual(values = c("#a79da1", "#79C9A2")) + # color code for sex categories  labs(title = "Survived by Age",  subtitle = "The Titanic Case",  caption = "Data from the Titanic dataset",  x = "\n Age", y = "Persons \n")+  theme(axis.text = element_text(size = 12),  axis.title = element_text(size = 12, face = "bold"),  plot.title = element_text(size = 14, hjust = 0.5, face = "bold")) ggplot(data = full1[!is.na(full[1:LT,]\)Age),{]},aes(x=Age,fill=Survived))+geom\_histogram(binwidth
= 3,position=``fill'')+ scale\_fill\_manual(values = c(``\#a79da1'',
``\#79C9A2'')) + labs(title = ``Proportion of Survived by Age'',
subtitle = ``The Titanic Case'', caption = ``Data from the Titanic
dataset'', x = ``\n Age'', y = ``Frequency \n'')+ theme(axis.text =
element\_text(size = 12), axis.title = element\_text(size = 12, face =
``bold''), plot.title = element\_text(size = 14, hjust = 0.5, face =
``bold'')) \# Children (under 15) and old people (80+) had a better
chance to survive. \# check the correlation of Fare versus Survivial
ggplot(data =
full{[}1:LT,{]},aes(x=Fare,fill=Survived))+geom\_histogram(binwidth =20,
position=``fill'')+ scale\_fill\_manual(values = c(``\#a79da1'',
``\#79C9A2'')) + \# color code for sex categories labs(title =
``Survived by Fare'', subtitle = ``The Titanic Case'', caption = ``Data
from the Titanic dataset'', x = ``\n Fare'', y = ``Persons \n'')+
theme(axis.text = element\_text(size = 12), axis.title =
element\_text(size = 12, face = ``bold''), plot.title =
element\_text(size = 14, hjust = 0.5, face = ``bold''))
full\(Fare[is.na(full\)Fare){]} \textless-
mean(full\(Fare,na.rm=T) sum(is.na(full\)Age)) \# seems like bigger fare
gave better chance to survive \# check the missing values for Age
sum(is.na(full\$Age))

\hypertarget{there-are-a-lot-of-missing-values-in-the-age-attribute-put-the-mean-instead-of-the-missing-values}{%
\section{There are a lot of missing values in the Age attribute, put the
mean instead of the missing
values}\label{there-are-a-lot-of-missing-values-in-the-age-attribute-put-the-mean-instead-of-the-missing-values}}

full\(Age[is.na(full\)Age){]} \textless-
mean(full\(Age,na.rm=T) sum(is.na(full\)Age)) \# check the influence of
a certain title of a passenger on the survival fact

full\(Title <- gsub('(.*, )|(\\..*)', '', full\)Name)
full\(Title[full\)Title == `Mlle'{]}\textless- `Miss'
full\(Title[full\)Title == `Ms'{]}\textless- `Miss'
full\(Title[full\)Title == `Mme'{]}\textless- `Mrs'
full\(Title[full\)Title == `Lady'{]}\textless- `Miss'
full\(Title[full\)Title == `Dona'{]}\textless- `Miss' officer\textless-
c(`Capt',`Col',`Don',`Dr',`Jonkheer',`Major',`Rev',`Sir',`the Countess')
full\(Title[full\)Title \%in\% officer{]}\textless-`Officer'

\hypertarget{convert-title-into-a-factor}{%
\section{convert Title into a
factor}\label{convert-title-into-a-factor}}

full\(Title<- as.factor(full\)Title)

\hypertarget{visualize-the-percentage-of-survived-vs-title}{%
\section{visualize the percentage of Survived vs
Title}\label{visualize-the-percentage-of-survived-vs-title}}

ggplot(data =
full{[}1:LT,{]},aes(x=Title,fill=Survived))+geom\_bar(position=``fill'')+
scale\_fill\_manual(values = c(``\#a79da1'', ``\#79C9A2'')) + \# color
code for sex categories labs(title = ``Survived by Title'', subtitle =
``The Titanic Case'', caption = ``Data from the Titanic dataset'', x =
``\n Title'', y = ``Percentage \n'')+ theme(axis.text =
element\_text(size = 12), axis.title = element\_text(size = 12, face =
``bold''), plot.title = element\_text(size = 14, hjust = 0.5, face =
``bold'')) Prediction At this time point, let's predict the chance of
survival as a function of the other attributes. Let's keep just the
correlated features: Pclass, Sex, Age, SibSp, Parch, Title and Fare.

The train dataset will be divided the train set into two sets: training
set (train1) and test set (train2) to be able to estimate the error of
the prediction.

\hypertarget{the-train-set-with-the-important-features}{%
\section{The train set with the important
features}\label{the-train-set-with-the-important-features}}

train\_im\textless-
full{[}1:LT,c(``Survived'',``Pclass'',``Sex'',``Age'',``Fare'',``SibSp'',``Parch'',``Title''){]}
ind\textless-sample(1:dim(train\_im){[}1{]},500) \# Sample of 500 out of
891 train1\textless-train\_im{[}ind,{]} \# The train set of the model
train2\textless-train\_im{[}-ind,{]} \# The test set of the model \# run
a logistic regression model \textless- glm(Survived
\textasciitilde.,family=binomial(link=`logit'),data=train1)
summary(model) \# It results as attributes SibSp, Parch and Fare are not
statisticaly significant. \# Let's look at the prediction of this model
on the test set (train2): pred.train \textless- predict(model,train2)
pred.train \textless- ifelse(pred.train \textgreater{} 0.5,1,0)

\hypertarget{mean-of-the-true-prediction}{%
\section{Mean of the true
prediction}\label{mean-of-the-true-prediction}}

mean(pred.train==train2\(Survived) # make a summary table of the prediction model t1<-table(pred.train,train2\)Survived)
t1 \# Presicion and recall of the model presicion\textless-
t1{[}1,1{]}/(sum(t1{[}1,{]})) recall\textless-
t1{[}1,1{]}/(sum(t1{[},1{]})) \# get the precision and recall parameters
presicion recall \# F1 score F1\textless-
2\emph{presicion}recall/(presicion+recall) F1 \# F1 score on the initial
test resulted at 0.879. That's pretty good \# Let's run it on the test
set:

test\_im\textless-full{[}LT+1:1309,c(``Pclass'',``Sex'',``Age'',``SibSp'',``Parch'',``Fare'',``Title''){]}

\hypertarget{make-at-the-prediction-of-this-model-on-the-test-set}{%
\section{make at the prediction of this model on the test
set:}\label{make-at-the-prediction-of-this-model-on-the-test-set}}

pred.test \textless- predict(model,test\_im){[}1:418{]} pred.test
\textless- ifelse(pred.test \textgreater{} 0.5,1,0)

\hypertarget{put-result-into-a-data-frame}{%
\section{put result into a data
frame}\label{put-result-into-a-data-frame}}

res\textless- data.frame(test\$PassengerId,pred.test)
names(res)\textless-c(``PassengerId'',``Survived'')

\hypertarget{put-the-prediction-result-into-a-.csv-file}{%
\section{put the prediction result into a .csv
file}\label{put-the-prediction-result-into-a-.csv-file}}

write.csv(res,file=``prediction.csv'',row.names = F) Building a
tree\ldots{} \# plant a tree and visualize it model\_dt\textless-
rpart(Survived \textasciitilde.,data=train1, method=``class'')
rpart.plot(model\_dt) \# make the prediction on the model pred.train.dt
\textless- predict(model\_dt,train2,type = ``class'')
mean(pred.train.dt==train2\(Survived) t2<-table(pred.train.dt,train2\)Survived)

presicion\_dt\textless- t2{[}1,1{]}/(sum(t2{[}1,{]}))
recall\_dt\textless- t2{[}1,1{]}/(sum(t2{[},1{]}))

\hypertarget{get-the-precision-and-recall}{%
\section{get the precision and
recall}\label{get-the-precision-and-recall}}

presicion\_dt recall\_dt \# get the F1 score F1\_dt\textless-
2\emph{presicion\_dt}recall\_dt/(presicion\_dt+recall\_dt) F1\_dt \# run
this model on the test set: pred.test.dt \textless-
predict(model\_dt,test\_im,type=``class''){[}1:418{]} res\_dt\textless-
data.frame(test\(PassengerId,pred.test.dt) names(res_dt)<-c("PassengerId","Survived") write.csv(res_dt,file="prediction_dt.csv",row.names = F) # make prediction on survival using a random forest model_rf<-randomForest(Survived~.,data=train1) # Let's look at the error plot(model_rf) # make the prediction on the model pred.train.rf <- predict(model_rf,train2) mean(pred.train.rf==train2\)Survived)
t1\textless-table(pred.train.rf,train2\(Survived) presicion<- t1[1,1]/(sum(t1[1,])) recall<- t1[1,1]/(sum(t1[,1])) presicion recall F1<- 2*presicion*recall/(presicion+recall) F1 # Let's run this model on the test set: pred.test.rf <- predict(model_rf,test_im)[1:418] res_rf<- data.frame(test\)PassengerId,pred.test.rf)
names(res\_rf)\textless-c(``PassengerId'',``Survived'')
write.csv(res\_rf,file=``submission\_rf.csv'',row.names = F) Conclusion
The mean of the right predictions:

decision tree method: 0.77837 random forest method: 0.77990 logistic
regression model: 0.7488

\begin{center}\rule{0.5\linewidth}{0.5pt}\end{center}

updated:

\begin{verbatim}
04/04/2023 - Made some cosmetics on some visuals (labels, theme...)
\end{verbatim}

The colors picked up considering some wide-spread advice on
visualization within the industry - to select the complementary colors
for better readability.

Used this resourse for the colour selection, actually.

\begin{verbatim}
09/04/2023 - Including pander library for certain table ra-shaping
\end{verbatim}

\end{document}
